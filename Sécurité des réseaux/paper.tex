\documentclass[journal, a4paper]{IEEEtran}

\usepackage[utf8x]{inputenc}
\usepackage[francais]{babel}

\usepackage{listings}

\usepackage{fancyvrb}

% some very useful LaTeX packages include:

%\usepackage{cite}      % Written by Donald Arseneau
                        % V1.6 and later of IEEEtran pre-defines the format
                        % of the cite.sty package \cite{} output to follow
                        % that of IEEE. Loading the cite package will
                        % result in citation numbers being automatically
                        % sorted and properly "ranged". i.e.,
                        % [1], [9], [2], [7], [5], [6]
                        % (without using cite.sty)
                        % will become:
                        % [1], [2], [5]--[7], [9] (using cite.sty)
                        % cite.sty's \cite will automatically add leading
                        % space, if needed. Use cite.sty's noadjust option
                        % (cite.sty V3.8 and later) if you want to turn this
                        % off. cite.sty is already installed on most LaTeX
                        % systems. The latest version can be obtained at:
                        % http://www.ctan.org/tex-archive/macros/latex/contrib/supported/cite/

\usepackage{graphicx}   % Written by David Carlisle and Sebastian Rahtz
                        % Required if you want graphics, photos, etc.
                        % graphicx.sty is already installed on most LaTeX
                        % systems. The latest version and documentation can
                        % be obtained at:
                        % http://www.ctan.org/tex-archive/macros/latex/required/graphics/
                        % Another good source of documentation is "Using
                        % Imported Graphics in LaTeX2e" by Keith Reckdahl
                        % which can be found as esplatex.ps and epslatex.pdf
                        % at: http://www.ctan.org/tex-archive/info/

%\usepackage{psfrag}    % Written by Craig Barratt, Michael C. Grant,
                        % and David Carlisle
                        % This package allows you to substitute LaTeX
                        % commands for text in imported EPS graphic files.
                        % In this way, LaTeX symbols can be placed into
                        % graphics that have been generated by other
                        % applications. You must use latex->dvips->ps2pdf
                        % workflow (not direct pdf output from pdflatex) if
                        % you wish to use this capability because it works
                        % via some PostScript tricks. Alternatively, the
                        % graphics could be processed as separate files via
                        % psfrag and dvips, then converted to PDF for
                        % inclusion in the main file which uses pdflatex.
                        % Docs are in "The PSfrag System" by Michael C. Grant
                        % and David Carlisle. There is also some information
                        % about using psfrag in "Using Imported Graphics in
                        % LaTeX2e" by Keith Reckdahl which documents the
                        % graphicx package (see above). The psfrag package
                        % and documentation can be obtained at:
                        % http://www.ctan.org/tex-archive/macros/latex/contrib/supported/psfrag/

%\usepackage{subfigure} % Written by Steven Douglas Cochran
                        % This package makes it easy to put subfigures
                        % in your figures. i.e., "figure 1a and 1b"
                        % Docs are in "Using Imported Graphics in LaTeX2e"
                        % by Keith Reckdahl which also documents the graphicx
                        % package (see above). subfigure.sty is already
                        % installed on most LaTeX systems. The latest version
                        % and documentation can be obtained at:
                        % http://www.ctan.org/tex-archive/macros/latex/contrib/supported/subfigure/

\usepackage{url}        % Written by Donald Arseneau
                        % Provides better support for handling and breaking
                        % URLs. url.sty is already installed on most LaTeX
                        % systems. The latest version can be obtained at:
                        % http://www.ctan.org/tex-archive/macros/latex/contrib/other/misc/
                        % Read the url.sty source comments for usage information.

%\usepackage{stfloats}  % Written by Sigitas Tolusis
                        % Gives LaTeX2e the ability to do double column
                        % floats at the bottom of the page as well as the top.
                        % (e.g., "\begin{figure*}[!b]" is not normally
                        % possible in LaTeX2e). This is an invasive package
                        % which rewrites many portions of the LaTeX2e output
                        % routines. It may not work with other packages that
                        % modify the LaTeX2e output routine and/or with other
                        % versions of LaTeX. The latest version and
                        % documentation can be obtained at:
                        % http://www.ctan.org/tex-archive/macros/latex/contrib/supported/sttools/
                        % Documentation is contained in the stfloats.sty
                        % comments as well as in the presfull.pdf file.
                        % Do not use the stfloats baselinefloat ability as
                        % IEEE does not allow \baselineskip to stretch.
                        % Authors submitting work to the IEEE should note
                        % that IEEE rarely uses double column equations and
                        % that authors should try to avoid such use.
                        % Do not be tempted to use the cuted.sty or
                        % midfloat.sty package (by the same author) as IEEE
                        % does not format its papers in such ways.

\usepackage{amsmath}    % From the American Mathematical Society
                        % A popular package that provides many helpful commands
                        % for dealing with mathematics. Note that the AMSmath
                        % package sets \interdisplaylinepenalty to 10000 thus
                        % preventing page breaks from occurring within multiline
                        % equations. Use:
%\interdisplaylinepenalty=2500
                        % after loading amsmath to restore such page breaks
                        % as IEEEtran.cls normally does. amsmath.sty is already
                        % installed on most LaTeX systems. The latest version
                        % and documentation can be obtained at:
                        % http://www.ctan.org/tex-archive/macros/latex/required/amslatex/math/



% Other popular packages for formatting tables and equations include:

%\usepackage{array}
% Frank Mittelbach's and David Carlisle's array.sty which improves the
% LaTeX2e array and tabular environments to provide better appearances and
% additional user controls. array.sty is already installed on most systems.
% The latest version and documentation can be obtained at:
% http://www.ctan.org/tex-archive/macros/latex/required/tools/

% V1.6 of IEEEtran contains the IEEEeqnarray family of commands that can
% be used to generate multiline equations as well as matrices, tables, etc.

% Also of notable interest:
% Scott Pakin's eqparbox package for creating (automatically sized) equal
% width boxes. Available:
% http://www.ctan.org/tex-archive/macros/latex/contrib/supported/eqparbox/

% *** Do not adjust lengths that control margins, column widths, etc. ***
% *** Do not use packages that alter fonts (such as pslatex).         ***
% There should be no need to do such things with IEEEtran.cls V1.6 and later.


% Your document starts here!
\begin{document}

% Define document title and author
	\title{Sécurité des réseaux}
	\author{Alexandre Kervadec
	\thanks{Professeur : A.Guermouche}}
	\markboth{Université de Bordeaux - Master 1 Informatique}{}
	\maketitle

% Write abstract here
\begin{abstract}
	Notes du cours de sécurité des réseaux de A.Guermouche
\end{abstract}

% Each section begins with a \section{title} command
\section{Les attaques}
	% \PARstart{}{} creates a tall first letter for this first paragraph
	\PARstart{O}{n} peut différencier une attaque d'une intrusion. Une attaque correspond à toute action compromettant la sécurité des informations. Une intrusion est la prise de contrôle partielle ou totale d'un système distant.
	
	\subsection{Description d'une attaque}
	
		\begin{itemize}
			\item \textbf{Recherche d'informations :} réseau, serveurs, routeurs, ...
			\item \textbf{Recherche de vulnérabilités :} OS, serveurs applicatifs, ...
			\item \textbf{Tentative d'exploitation des vulnérabilités :} à distance puis localement
			\item \textbf{Installation de backdoor}
			\item \textbf{Installation de sniffer}
			\item \textbf{Suppression des traces}
			\item \textbf{Attaque par déni de service} 
		\end{itemize}
		
	\subsection{But des attaques}
	
		\textbf{Interruption :} vise la disponibilité des informations (DoS, ...).
		
		\begin{figure}[!hbt]
			\begin{center}
				\includegraphics[width=\columnwidth]{img/but_inter.png}
				\label{fig:but_inter}
			\end{center}
		\end{figure}
		\textbf{Interception :} vise la confidentialité des informations (capture de contenu, analyse de traffic, ...).
		
		\begin{figure}[!hbt]
			\begin{center}
				\includegraphics[width=\columnwidth]{img/but_intercep.png}
				\label{fig:but_intercep}
			\end{center}
		\end{figure}
		\newpage
		\textbf{Modification :} vise l'intégrité des informations (modification, rejeu, ...).
		
		\begin{figure}[!hbt]
			\begin{center}
				\includegraphics[width=\columnwidth]{img/but_modif.png}
				\label{fig:but_modif}
			\end{center}
		\end{figure}
		\textbf{Fabrication :} vise l'authenticité des informations (mascarade, ...).
		
		\begin{figure}[!hbt]
			\begin{center}
				\includegraphics[width=\columnwidth]{img/but_fab.png}
				\label{fig:but_fab}
			\end{center}
		\end{figure}
		
	\subsection{Technique de recherche d'information}
		\begin{itemize}
			\item Recherche d'informations publiques : DNS, \verb|whois|, ...
			\item Découverte du réseau et du filtrage IP : \verb|traceroute|, \verb|ping|, \verb|hping|, \verb|netcat|, ...
			\item Découverte des systèmes d'exploitation : \verb|nessus|, \verb|nmap|, \verb|xprobe|, \verb|queso|, ...
			\item Découverte de services ouverts : \verb|nmap|, \verb|udp-scan|, \verb|nessus|, ...
			\item Découverte des versions logicielles : \verb|telnet|, \verb|netcat|, ...
		\end{itemize}
		
	\subsection{Exemple : découverte des machines via DNS}
		Interrogation du DNS avec \verb$dig$ :
		\begin{itemize}
			\item serveur de mail (champ MX), serveur DNS (champ NS)
			\item résolution inverse sur toutes les adresses (\textsl{peu discret}) : \verb$dig -x$
			\item transfert de zone (\textsl{pas toujours autorisé}) : \verb$dig server axfr zone.$
		\end{itemize}
		\begin{Verbatim}[fontsize=\small]
>dig labri.fr. MX
; <<>> DiG 9.4.1-P1 <<>> labri.fr. MX
;; global options: printcmd
;; Got answer:
;; ->>HEADER<<- opcode: QUERY, status:
NOERROR, id: 22464
...
;; QUESTION SECTION:
;labri.fr. IN MX
;; ANSWER SECTION:
labri.fr. 28800 IN MX 10 iona.labri.fr.
...
		\end{Verbatim}
		
		
		

	
% Your document ends here!
\end{document}